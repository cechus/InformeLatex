\documentclass[letterpaper]{article}
\usepackage[utf8]{inputenc}
\usepackage[english, spanish]{babel}
\usepackage{amsmath}
\usepackage{tikz}
\usepackage{epigraph}
\usepackage{lipsum}
\usepackage[hyphens]{url}
\usepackage[breaklinks,colorlinks=true,linkcolor=red,
citecolor=red, urlcolor=blue]{hyperref}
\renewcommand\epigraphflush{flushright}
\renewcommand\epigraphsize{\normalsize}
\setlength{\parindent}{1cm}
\renewcommand{\labelitemi}{$-$}
\setlength\epigraphwidth{0.7\textwidth}
%\usepackage[T1]{fontenc}
%\usepackage{inconsolata}
%\renewcommand*\familydefault{\ttdefault}


\definecolor{titlepagecolor}{cmyk}{1,.60,0,.40}

\DeclareFixedFont{\titlefont}{T1}{ppl}{b}{it}{0.5in}

\makeatletter                       
\def\printauthor{%                  
    {\large \@author}}              
\makeatother
\author{%
    Jhonatan I. Castro Rocabado \\
    \texttt{}\vspace{20pt} \\
    Oscar R. Choque Huayta\\
    \texttt{}
    }

% The following code is borrowed from: http://tex.stackexchange.com/a/86310/10898

\newcommand\titlepagedecoration{%
\begin{tikzpicture}[remember picture,overlay,shorten >= -10pt]

\coordinate (aux1) at ([yshift=-15pt]current page.north east);
\coordinate (aux2) at ([yshift=-410pt]current page.north east);
\coordinate (aux3) at ([xshift=-4.5cm]current page.north east);
\coordinate (aux4) at ([yshift=-150pt]current page.north east);

\begin{scope}[titlepagecolor!40,line width=12pt,rounded corners=12pt]
\draw
  (aux1) -- coordinate (a)
  ++(225:5) --
  ++(-45:5.1) coordinate (b);
\draw[shorten <= -10pt]
  (aux3) --
  (a) --
  (aux1);
\draw[opacity=0.6,titlepagecolor,shorten <= -10pt]
  (b) --
  ++(225:2.2) --
  ++(-45:2.2);
\end{scope}
\draw[titlepagecolor,line width=8pt,rounded corners=8pt,shorten <= -10pt]
  (aux4) --
  ++(225:0.8) --
  ++(-45:0.8);
\begin{scope}[titlepagecolor!70,line width=6pt,rounded corners=8pt]
\draw[shorten <= -10pt]
  (aux2) --
  ++(225:3) coordinate[pos=0.45] (c) --
  ++(-45:3.1);
\draw
  (aux2) --
  (c) --
  ++(135:2.5) --
  ++(45:2.5) --
  ++(-45:2.5) coordinate[pos=0.3] (d);   
\draw 
  (d) -- +(45:1);
\end{scope}
\end{tikzpicture}%
}

\begin{document}
\begin{titlepage}

\noindent
\titlefont Documento de\\de requerimientos \\
para \\Se Busca
\par
\null\vfill
\vspace*{1cm}
\noindent
\hfill
\begin{minipage}{0.35\linewidth}
    \begin{flushright}
        \printauthor
    \end{flushright}
\end{minipage}
%
\begin{minipage}{0.02\linewidth}
    \rule{1pt}{125pt}
\end{minipage}
\titlepagedecoration
\end{titlepage}
\tableofcontents
\newpage
\begin{center}
\begin{Huge}
\textbf{Se Busca\\
Sistema de control para contrarrestar la Trata y
el Tráfico de Personas}
\end{Huge}
\end{center}
\section*{Miembros del equipo:}
\begin{itemize}
\item[$-$] Jhonatan Ismael Castro Rocabado.
\item[$-$] Oscar Ramiro Choque Huayta.
\end{itemize}
\section{Placificacíon Inicial}
\subsection{Solicitud de proyecto}
El proyecto fue solicitado en la materia de INF-272, \"Taller de Base de
Datos\", con el fin de aplicar los conocimientos de desarrollo de software
adquiridos hasta ahora.
\subsection{El porqué de la solicitud del proyecto}
\subsubsection*{La problemática}
La problemática que se aborda con el proyecto es el "Trata Tráfico de personas" como indica el título. Los últimos tiempos este ha sido un problema
constante en la sociedad boliviana y en otros países.\\
La forma en la que nuestro proyecto piensa atacar a este problema es
brindando información geoposicional de los lugares donde las personas son
vistas por última vez y brindando una plataforma en la que se pueda
denunciar a todas las personas desaparecidas.
\subsection{Evaluación de la información deseada }
\subsubsection*{Los requerimientos}
El sistema será diseñado para retroalimentarse de información
automáticamente, es decir, los datos que utilizará el sistema serán los mismos
que los usuarios aporten.
\subsection{Situación actual}
Actualmente no existe un sistema informático que ayude en esta
problemática, esta es la razón por la cual se decidió a trabajar en este.
\subsection{Estructura de División del trabajo}
El desarrollo del Software se dividirá en las siguientes etapas y áreas.
\begin{itemize}
\item Ingeniería de requerimientos.
\item Diseño y análisis del sistema.
\item Desarrollo del “backend”.
\item Desarrollo del “frontend”.
\item Administración del servidor.
\item Mantenimiento al sistema
\end{itemize}
\subsection{Conformación de equipo humano}
El equipo estará conformado por los estudiantes de la materia INF-272, "Taller de Base de Datos":
\begin{itemize}
\item Jhonatan Castro.
\item Oscar Choque.
\end{itemize}
\subsection{Estimación del plazo de entrega y precio}
La estimación de tiempo se hará cuando se tengan bien definidos los
módulos que serán desarrollados para el sistema.
\\
Como este es un proyecto con una licencia libre, inicialmente no se
calculó el precio, pero esto se hará cuándo se tenga la información de la cantidad de horas que costará hacer el análisis, desarrollo y mantenimiento del sistema. A esto se multiplicará la experiencia y calidad de trabajo de los miembros del equipo.
\section{Herramientas que se utilizarán}
\begin{itemize}
\item Para el desarrollo del software se utilizará el Framework “Laravel” desarrollado en el lenguaje PHP. Este cuenta con una licencia libre. \\ \url{www.laravel.com}
\item Se hará el versionamiento con la herramienta Git y la plataforma Github, acá la dirección del proyecto:
\url{https://github.com/jhtan/sebusca}
\item Se controlará el avance de los miembros y se aplicará la metodología SCRUM utilizando la plataforma web también libre, Taiga.
\url{https://taiga.io/}
\end{itemize}
\section{SCRUM}
\subsection{Primer Sprint}
\subsubsection*{Objetivos:}
Análisis y diseño de "Se Busca"
\begin{table}[h!]
\centering
\begin{tabular}{|l|c|} \hline
\textbf{Historias de Usuario}& \textbf{Horas de Esfuerzo} \\\hline
Diseño de Base de Datos &  \\\hline
Diseño de ñas pantallas principales &  \\\hline
\end{tabular}
\end{table}
\subsection{Segundo Sprint}
\subsubsection*{Objetivos: }
\begin{table}[h!]
\centering
\begin{tabular}{|c|c|} \hline
\textbf{Historias de Usuario}& \textbf{Horas de Esfuerzo} \\\hline
 &  \\\hline
 &  \\\hline
\end{tabular}
\end{table}
\subsection{Backlog}
\subsubsection*{Objetivos:}
\begin{table}[h!]
\centering
\begin{tabular}{|c|c|} \hline
\textbf{ }& \textbf{ } \\\hline
 &  \\
 \hline
 &  \\\hline
\end{tabular}
\end{table}
\end{document}